Tagsets are compilations of the morphosyntactic roles which try to encompass a full representation of the different morphological behaviours and syntactic functions words can occupy in a certain language. 
As stated before the delimitation of these categories is not strictly defined and therefore a multitude of different tagsets has been introduced. 

Because of the ongoing discussions and the frequently indistinct boundaries of grammatical categories, tagsets aim to be pragmatic, since a pure theoretical, all-encompassing distinction of different POSs is not yet achieved and will likely not be any time soon \citep{westpfahl2020pos}. 
Additionally the slight alterations that occur constantly in natural languages impedes the realisation of a definitive discrimination.

Among the most recognized tagsets is the Brown tagset which is based on the 82 categories used on the Brown corpus. The rationale for the tagging procedure was initially introduced in \citealt{greene1971automatic}, but was extended by numerous contributors \citep{brown1979manual}. 

Another prominent tagset is CLAWS, Constituent Likelihood Automatic Word-tagging System, introduced by the Lancaster University which encompasses a varying number of different POS-tags depending on the version of its tagset. It has been continuously developed since the early 1980s \citep{garside1987claws}. Currently 8 versions are published.

The English Penn Treebank tagset consists of 36 POS-tags and 12 tags specifically for punctuation purposes. After the compilation of the roughly 4.5 million word Penn corpus by the University of Pennsylvania in 1989, it has been annotated in the following three years with its corresponding tagset \citep{marcinkiewicz1994building}.

In this thesis the tagset of the Universal Dependencies (UD) framework with its 17 tags will be used. It has become widely accepted as one of the most important tagsets, since it has treebanks for over 100 languages \citep{ud2022opencom}. It is an open community effort. The tags in it are:
\begin{description}
\item[ADJ] Adjective: noun modifiers describing properties
\item[ADV] Adverb: verb modifiers of time, place, manner
\item[NOUN] words for persons, places, things, etc.
\item[VERB] words for actions and processes
\item[PROPN] Proper noun: name of a person, organization, place, etc..
\item[INTJ] Interjection: exclamation, greeting, yes/no response, etc.
\item[ADP] Adposition (Preposition/Postposition): marks a noun’s spacial, temporal, or other relation
\item[AUX] Auxiliary: helping verb marking tense, aspect, mood, etc.
\item[CCONJ] Coordinating Conjunction: joins two phrases/clauses
\item[DET] Determiner: marks noun phrase properties
\item[NUM] Numeral
\item[PART] Particle: a preposition-like form used together with a verb
\item[PRON] Pronoun: a shorthand for referring to an entity or event
\item[SCONJ] Subordinating Conjunction: joins a main clause with a
subordinate clause such as a sentential complement
\item[PUNCT] Punctuation
\item[SYM] Symbols like \$ or emoji
\item[X] Other
\end{description}














