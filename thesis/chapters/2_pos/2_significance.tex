A variety of natural language processing tasks utilize the information provided by tags concerning the POS of tokens. 
While the concrete instantiation is seldom considered a relevant piece of information in the finalized output of tasks such as machine translation or information retrieval (IR), it can help to disambiguate the semantic nature of a word during parsing. 
For tokens that belong to several morphosyntactic classes inhibit an ambiguity regarding their meaning \citep{embedding2020pilehvar}.

The filtering of nouns of a corpus may provide valuable insight for the Information-Retrieval task. 

Machine translation algorithms often use POS-tags as one source of information \citep{westpfahl2020pos}.

A POS-tag is also relevant to disambiguate the cases in which varying POS-instantiations would be pronounced differently. 
The different pronunciations of 'object' come into place whether it is considered a noun or a verb where in the case of being a noun the first syllable would be stressed and in case of being a verb the second syllable.
'Content' has an emphasis on the first syllable when instantiated as a noun or on the second if considered an adjective. 

These two examples should showcase how the Part-of-Speech-tag can be relevant for producing a more accurate speech recognition system or help to choose the correct pronunciation for speech synthesis systems \citep{jurafsky2021}. 

A correct classification also provides more subtle cues considering the stem of a word or the probability of words which could appear in its vicinity.
