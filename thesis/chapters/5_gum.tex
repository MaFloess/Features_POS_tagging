The Georgetown University Multilayer Corpus (GUM) is an open source collection of texts of multiple types which get annotated extensively \citep{ud2022opencom}. At the Georgetown University students collect and expand the corpus as a part of their curriculum by adding layers of analysis to a text chosen from openly available sources \citep{Zeldes2017}.

The POS-tags are manually annotated using the Stanford Typed Dependencies \citep{de2008stanford} and are converted to the Universal Dependencies tagset using the DepEdit tool \citep{ud2022opencom}. Afterwards the tags are corrected manually using the Universal Dependencies guidelines, to ensure the quality of the annotations.

The text types used in the data are meant to represent a variety of communicative purposes and stem from openly available sources to prevent restrictive licenses from interfering with the process of annotating and publishing.
Sources include Wikinews, Wikipedia and reddit, while the text types are divided in categories in such as interviews, news articles, biographies and fiction \citep{Zeldes2017}.

The fact of being open source helps making the analysis reproducible. Being compiled by the Georgetown University ensures that the data follows institutional guidelines and quality standards, while the corpus size of roughly 7 thousand sentences allows for reasonable computability speeds on machines with an average computing capability. Therefore the choice of data for this thesis has been the GUM dataset.